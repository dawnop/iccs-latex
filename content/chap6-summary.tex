\section{Conclusion}

%Currently, the analysis of programming language design is still relatively rare compared to the mainstream research direction of PLT. In particular, a systematic approach to the study of programming language design is still lacking. In this paper, we present a relatively comprehensive analysis of programming language design through several aspects affecting programming language applications with the help of several specific MPLs.
%
%In Chapter 2, we discuss the definition of MPL. Based on that, the popular programming languages on IEEE Spectrum with MPL features are selected. Then we discuss the factors that a good programming language design should have. In Chapter 3, we stand in the perspective of programming paradigms, give a method to judge the degree of MPL programming paradigm support, and do an analysis of the degree of support of the selected MPLs. This is followed by an research of the relationship between the strength of programming paradigm support and programming language applications. In Chapter 4, we discuss the type system. The type system of the selected MPLs is discussed based on L Cardelli's type system definition. In Chapter 5, we discuss the performance of programming languages through three benchmark tests in terms of memory allocation and compilation methods. Then, we analyze the data from the benchmark tests from the perspective of programming language implementations and the application scenarios of these languages to explain the reasons for the performance of these MPLs.
%
%Programming paradigms, type systems, and performance are three interconnected parts that cut across multiple aspects of programming languages that need to be considered from design to implementation. Each of these three branches has a significant impact on the design of programming languages, influencing their expressiveness, maintainability, reliability, and performance, the most important metrics of programming languages. In fact we can obtain that the process of theoretical and practical development of programming languages is always dynamic, as is the design and implementation of programming languages. Programming paradigms, type systems, and performance are three interconnected parts that cut across multiple aspects of programming languages that need to be considered from design to implementation. Each of these three branches has a significant impact on the design of programming languages, influencing their expressiveness, maintainability, reliability, and performance, the most important metrics of programming languages. In fact we can obtain that the process of theoretical and practical development of programming languages is always dynamic, as is the design and implementation of programming languages. This paper dynamically discusses programming language design based on MPL, which would be a preferable research idea from a programming language design and analysis perspective.


Currently, the research on programming language design is still relatively rare compared to the mainstream research direction of PLT which is the implementation of programming language. In this paper, we present a relatively comprehensive analysis of the design of multiple MPLs through several aspects affecting their applications.

We first select some popular programming languages on IEEE Spectrum with MPL features after discussing the definition of MPL. Then we discuss the features that a well-designed programming language should have. Next, we stand in the perspective of programming paradigms to give a method to judge the degree of MPLs’ paradigm support, and do an analysis of the selected MPL, followed by the relationship between the degree of programming paradigm support and programming language applications. Furthermore, the type systems of the selected MPLs are discussed based on L. Cardelli’s definition of type systems. Finally, we evaluate the performance of the selected MPLs through three benchmark tests and explain the reasons for the performance from the perspective of programming language implementations and their application scenarios.

Programming paradigms, type systems, and performance are three critical interconnected parts that have a significant impact on the design of programming languages, influencing their most important metrics, namely expressiveness, maintainability, reliability, and performance. In fact, the theory and practice of programming languages are always dynamically developing, as are their design and implementation. This paper comprehensively discusses the design of certain MPLs, which would be a practical reference and a valuable inspiration for programming language developers and programmers.
